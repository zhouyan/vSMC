\chapter{Configuration macros}
\label{chap:Configuration macros}

The library has a few configuration macros. All macros have a default value if
left undefined, and can be overwritten by defining them with proper values
before including any of the library headers. In this appendix, we detail all
the configuration macros used by the library.

There are three types of configuration macros. The first type has a prefix
\verb|VSMC_HAS_|. These macros specify a certain feature or third-party library
is available. The second type has a prefix \verb|VSMC_USE_|. These macros
specify that a certain feature or third-party library can be used, if
available. These two kinds of macros shall only be defined to integral values,
with zero as false and all other values as true. The third type can be defined
to arbitrary, and usually affects implementation choices made by the library.

A concrete example of the relations among these three types of configuration
macros is the memory allocation functions used by the library. The details is
in section~\ref{sec:Aligned memory allocation}. In short, they library has
three implementations for aligned memory allocation. The preferred method is to
use the \verb|scalable_aligned_malloc| from \tbb. It is defined through a class
called \verb|AlignedMemoryTBB|. This class is only defined if and only if
\verb|VSMC_HAS_TBB| is defined to a non-zero, in which case, the runtime
library \verb|tbbmalloc| will need to be linked as well (\verb|-ltbbmalloc| on
\unix-alike systems). However, if it is desirable to define this class such
that the user can use it with \stl containers, but it is undesirable for the
library itself to use it everywhere, then one can further define the macro
\verb|VSMC_USE_TBB_MALLOC| to zero. In this case, the class is still defined,
but it will not be used by the library. Instead the system will now prefer
operating system dependent memory allocation functions such as
\verb|posix_memalign|. Now, suppose this is still undesirable, and one want to
use a memory allocation library not directly supported by the library. In this
case, one can implement a aligned memory class (see section~\ref{sec:Aligned
  memory allocation} for details), say \verb|AlignedMemoryUser|. And then define
the macro \verb|VSMC_ALIGNED_MEOMRY_TYPE| to this class.

In summary, the \verb|VSMC_HAS_| macro affects what will be \emph{defined}
by the library. The \verb|VSMC_USE_| macro affects what will be \emph{used}
by the library. And other macros give direct control of the library's
implementations. In most situations, it is sufficient to use the first two to
configure the features of the library. The third kind is for more advanced
users.

\section{System dependent macros}
\label{sec:System dependent macros}

The following macros have default values that depend on the compiler, operating
system and \cpu types.
